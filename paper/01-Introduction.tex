\chapter{Introduction}
\label{chapt:Introduction}
In today's complex and interconnected world, the modeling and simulation of concurrent systems play a pivotal role in understanding and analyzing various phenomena, ranging from distributed algorithms to communication protocols. Synchronous Concurrent Constraint Datalog (SCCD) has emerged as a powerful formalism for expressing and reasoning about concurrent systems, offering expressive capabilities to capture intricate interactions among components. On the other hand, the PyDEVS (Python-based Discrete Event System Specification) framework provides a versatile platform for discrete event simulation, facilitating the modeling and analysis of dynamic systems.

However, despite their individual strengths, SCCD and PyDEVS represent distinct simulation paradigms with differences in modeling semantics, computational approaches, and tooling ecosystems. This dichotomy poses challenges for researchers and practitioners seeking to leverage the advantages of both SCCD and PyDEVS for modeling and simulating concurrent systems.

This master's thesis aims to bridge this gap by exploring the translation of SCCD models into the PyDEVS framework, thereby facilitating the integration of SCCD-based approaches into the PyDEVS ecosystem. By enabling the simulation of SCCD models using PyDEVS, this research seeks to extend the reach of SCCD-based methodologies to a broader audience while harnessing the capabilities of PyDEVS for efficient and scalable simulation.

In this introduction, we provide an overview of SCCD and PyDEVS, highlighting their respective features, strengths, and applications. We also outline the motivation behind this research, discussing the potential benefits of integrating SCCD into PyDEVS and the challenges inherent in reconciling the differences between these simulation paradigms. Furthermore, we present an outline of the thesis, detailing the structure and organization of subsequent chapters, which include an analysis of SCCD and PyDEVS, a methodology for SCCD-to-PyDEVS translation, case studies demonstrating the translation approach, and discussions on the implications and future directions of this research.

Overall, this thesis aims to contribute to the advancement of discrete event simulation by bridging simulation paradigms and facilitating the seamless integration of SCCD models into the PyDEVS framework, thereby fostering innovation and exploration in concurrent system modeling and analysis.

\section{Definitions}
\begin{enumerate}
    \item \textbf{Statecharts and class diagrams (SCCD):} A language that combines the Statecharts language with Class Diagrams. It allows users to model complex, timed, autonomous, reactive, dynamic-structure systems.
    
    \item \textbf{PyDEVS:} Python-based Discrete Event System Specification, a framework for discrete event simulation that provides a platform for modeling and analyzing dynamic systems using the DEVS (Discrete Event System Specification) formalism.
            
    \item \textbf{Discrete Event Simulation:} A simulation technique used to model systems where events occur at distinct points in time, with changes in system state triggered by these events.
    
    \item \textbf{Translation:} The process of converting models or specifications from one formalism or representation to another while preserving essential characteristics and semantics.
    
    \item \textbf{Modeling Paradigm:} A set of principles, concepts, and methodologies used for constructing models to represent real-world systems or phenomena.
    
    \item \textbf{Semantic Differences:} Variations in meaning or interpretation between different formalisms or modeling approaches, which may require adjustments or transformations during the translation process.
    
    \item \textbf{Computational Efficiency:} The ability of a simulation framework or algorithm to execute simulations with minimal computational resources, such as time and memory.
    
    \item \textbf{Statechart:}

    \item \textbf{SCXML:}
  \end{enumerate}

