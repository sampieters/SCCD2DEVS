\chapter{Methodology}
\label{chapt:Methodology}
The methodology employed in this thesis outlines the systematic approach taken to translate SCCD into PyDEVS. This section provides an overview of the strategies, tools, and processes utilized to achieve the translation objectives. 
The translation involves converting models represented in one formalism into another while preserving their essential behavioral characteristics. Such translation is critical for interoperability, model verification, and simulation 
interoperability between systems designed using different modeling paradigms.

We also present the overarching methodology for translating SCCD to PyDEVS, detailing the steps involved, challenges encountered, and the rationale behind the chosen approaches. Additionally, we discuss alternative 
approaches considered during the development process, providing insights into why certain strategies were pursued while others were discarded.

\section{Translation Tools}

The translation of SCCD to DEVS involves the use of specialized tools and techniques to facilitate the conversion process. In this section, we discuss the evolution of the translation tools employed in this research, from initial 
parsing methods to the adoption of a visitor pattern for seamless translation.

\subsection{Parsing line by line}
Initially, in the endeavor to translate SCCD to DEVS, the SCXML file was parsed meticulously, line by line. However, as we delved deeper into the process, we encountered challenges that made this approach increasingly cumbersome. 
The complexity of the SCXML structure made it difficult to extract the necessary information efficiently, leading to inefficiencies and potential errors in the translation process. Recognizing the need for a more robust and 
structured approach, we sought alternative solutions to streamline our workflow.

\subsection{Visitor Pattern}
Upon further investigation, it was discovered that the SCCD compiler (TODO: reference) already implemented a visitor pattern for translating SCXML files into Python and/or Java representations. Leveraging this existing pattern 
proved to be a significant breakthrough in the translation process. Not only allows the pattern for the traversal of complex data structures without modifying the structure itself, but it also allows for an easy integration of 
a new language or platform. 

The compiler will generate an Abstrax Syntax Tree


\section{Considered Translation Approaches}

\subsection{First approach: Implementing everything into one AtomicDEVS}

\subsection{Second approach: Setting every component to AtomicDEVS}

The first approach was to set every possible component (Controller, ObjectManager and classes) to an AtomicDEVS model. This proved to be impossible with the standard
atomicDEVS because in SCCD, statecharts can be created at runtime. The classic AtomicDEVS does not allow to create AtomicDEVS objects at runtime and thus mapping statecharts
to a corresponding AtomicDEVS is impossible.


