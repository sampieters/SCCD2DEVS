\clearpage 
\phantomsection 
\addcontentsline{toc}{chapter}{Abstract}
\chapter*{Abstract}
This master's thesis explores the translation of Synchronous Concurrent Constraint Datalog (SCCD) models into the PyDEVS (Python-based Discrete Event System Specification) framework, investigating its implications and applications. SCCD, a formalism for modeling concurrent systems, offers powerful expressive capabilities for describing complex interactions. PyDEVS, on the other hand, provides a versatile platform for discrete event simulation.

The primary objective of this research is to bridge the gap between these two simulation paradigms, facilitating the integration of SCCD models into the PyDEVS ecosystem. This translation enables the utilization of SCCD models within PyDEVS for simulation and analysis, thereby extending the reach of SCCD-based approaches to a broader audience of researchers and practitioners.

The thesis first presents an in-depth analysis of SCCD and PyDEVS, highlighting their respective features, strengths, and weaknesses. Subsequently, it proposes a systematic methodology for translating SCCD models into PyDEVS specifications, addressing challenges such as semantic differences, modeling paradigms, and computational efficiency.

To demonstrate the utility of this translation approach, the thesis showcases several case studies where SCCD models are successfully translated into PyDEVS and simulated using the PyDEVS framework. These case studies encompass diverse application domains, including concurrent systems, distributed algorithms, and communication protocols.

Furthermore, the thesis discusses the practical implications and benefits of integrating SCCD into PyDEVS, such as enhanced scalability, interoperability with existing simulation tools, and the ability to leverage PyDEVS's rich ecosystem of libraries and tools for simulation analysis.

Overall, this research contributes to advancing the field of discrete event simulation by facilitating the seamless integration of SCCD models into the PyDEVS framework, opening up new avenues for research and applications in diverse domains requiring concurrent system modeling and analysis.